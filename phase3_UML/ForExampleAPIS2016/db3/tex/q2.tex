\documentclass[]{article}
\usepackage{fullpage}
\usepackage{enumerate}
\usepackage[T1,T2A]{fontenc}
\usepackage[utf8]{inputenc}
\usepackage[bulgarian]{babel}

\begin{document}
\title{Информационна система на агенция за недвижими имоти}
\author{Екип $\pi \approx 3.1$}

\maketitle

\begin{center}

\begin{tabular}{|l|l|}
\hline
Дата и час	& 2016-03-18, от 16:00 до 17:30 \\ \hline
Място		& ФМИ, зала 01 \\ \hline
Клиент		&  \\ \hline
			& Борислав Арнаудов - собственик на стартираща агенция за недвижими имоти \\ \hline
Изпълнител	&  \\ \hline
			& Георги Димов \\ \hline
			& Цветан Цветанов \\ \hline
			& Антон Дудов \\ \hline
			& Венцислав Конов \\ \hline
			& Александър Танков \\ \hline
			& Красимир Тренчев \\ \hline
			& Александър Велин \\ \hline
			& Анджелика Туджарска \\ \hline
			& Александър Бранев \\ \hline
			& Мартин Стоев \\ \hline
\end{tabular}

\end{center}

\begin{enumerate}[I.]
	\item {Обща представа за системата
	
Трябва да се проектира информационна система за стартираща агенция за недвижими имоти.

Целта на информационната система е да осигури уеб интерфейс, чрез който брокерите могат да управляват своите обяви за недвижими имоти, а клиентите на агенцията (наематели и купувачи) да
имат гъвкав и удобен интерфейс за търсене на обяви за имоти по различни критерии, както и да инициират комуникация с брокера. Връзката между продавачите и наемодателите с брокер ще се осъществява извън рамките на системата.

Всички счетоводни и финансови процеси в агенцията не са предмет на информационната система.
	}
	
	\item {Функционалност на системата
		\begin{enumerate}[1.]
			\item {Достъп и акаунти

				Системата трябва да поддържа както анонимен (публичен) достъп, така и достъп на потребители, които имат създаден акаунт в системата. Всеки акаунт има следните данни:

				\begin{itemize}
					\item username (задължителен)
					\item парола (задължителен)
					\item email (задължителен)
					\item име (задължителен)
					\item телефон
					\item снимка
				\end{itemize}
			}
			\item {Роли			

				Системата трябва да поддържа следните роли:
	
				\begin{itemize}
					\item {нерегистриран потребител - има права да:
						\begin{itemize}
							\item търси и разглежда обяви в сайта
							\item споделя обяви в социалните мрежи
							\item си регистрира акаунт, с потвърждение по email, за да се гарантира вярност на вписания адрес
							\item използва контакт формата за изпращане на съобщение към брокерите
							\item използва чат системата, като не му се запазва log на разговора
						\end{itemize}
					}
					\item {регистриран потребител - има права да:
						\begin{itemize}
							\item прави всички неща, които може нерегистриран потребител, освен регистрация на акаунт
							\item използва чат системата, като се запазва log на разговора
							\item ``запазва'' обяви, които е харесал във профила си
							\item променя личните си данни
							\item поиска ресет на паролата си (новата му се изпраща на мейла)
							\item подаде заявка, че иска да стане брокер
						\end{itemize}
					}
					\item {брокер - има права да:
						\begin{itemize}
							\item прави всички неща, които може регистриран потребител
							\item създава обяви за имоти
							\item редактира и премахва своите обяви за имоти
							\item променя състоянието на своите обяви (активни, неактивни, създадени)
							\item променя статуса на своя обява (нормална, VIP)
							\item комуникира през чат системата с други потребители, ако те са инициирали комуникацията
							\item получава съобщения през контакт формата
							\item вижда точният адрес на имота в своите обяви
							\item вижда контактната информация за собствениците на имотите в своите обяви
						\end{itemize}
					}
					\item {администратор (вграден в системата сервизен акаунт, единствен):
						\begin{itemize}
							\item има права да променя личните си данни
							\item има право да разглежда личните данни на потребителите (username, email, име, телефон, снимка)
							\item няма право да променя личните данни на потребителите (username, парола, email, име, телефон, снимка)
							\item има право да одобрява заявки на регистрирани потребители за промяна на статуса им към 'брокер'
							\item има право да премахва статуса 'брокер' от акаунти
							\item има право да премахва акаунти от системата, без сервизните такива
							\item има право да премахва и реадктира обяви
							\item получава копие от съобщенията, изпратени през контакт формата
						\end{itemize}
					}
					\item одитор (вграден в системата сервизен акаунт, единствен) -- има права само да чете одит лога
				\end{itemize}
			}
			\item {Обяви
			
				Обявите се създават, редактират и премахват от брокер. Администраторът има право да редактира и премахва обявите на всички брокери.

				Обявите могат да бъдат активни или неактивни, като обикновенните потребители (нерегистрирани и регистрирани) виждат само активните обяви, докато неактивните обяви са видими само за брокерите и администратора.

				Обявите могат да бъдат нормални или VIP. Този статус може да се променя само ръчно от брокерите (и администратора).
				
				Обявите съдържат следните данни:
				\begin{itemize}
					\item идентификатор на брокер
					\item активна/неактивна обява 
					\item нормална/VIP обява
					\item {характеристики на имота:
						\begin{itemize}
							\item местоположение - град, квартал
							\item детайлно местоположение - град, квартал, улица, номер, и ако е нужно вход, етаж, апартамент
							\item площ в квадратни метри
							\item тип на цената (месечен наем/цена за закупуване), валута (BGN, USD, EUR), цена
							\item тип на имота - апартамент, къща, гараж, парцел
							\item {специфични характеристики според типа:
								\begin{itemize}
									\item апартамент -- тип (едностаен, двустаен, тристаен, многостаен, мезонет, студио), етаж, изложение, година и тип на строителството, прилежащи имоти (мазета, гаражи) и общи части, обзавеждане, интернет, ТЕЦ, СОТ, телефон, ток, вода
									\item къща -- застроена и дворна площ, брой етажи, прилежащи имоти (мазета, гаражи),	обзавеждане, интернет, ТЕЦ, СОТ, телефон, ток, вода
									\item гараж -- ток, вода
									\item парцел -- тип(в регулация, земеделска земя), ток, вода
								\end{itemize}
							}
						\end{itemize}
					}
					\item текстово поле -- свободно описание на обявата, до 5 KB. Полето трябва да позволява въвеждането на символи за нов ред
					\item снимки и скици -- binary формат, до 20 броя общо, всяка с размер до 1 MB 
					\item географски данни, описващи точната локация (GPS координати)
					\item информация за наемодателя/продавача -- имена, телефон, email, текст
				\end{itemize}
				
				Информацията за наемодателя/продавача и детайлното местоположение е достъпна само за брокерите и администратора.
			}
			
			\item {Визуализация на обява

				При визуализация на детайлите на дадена обява, системата трябва да:
				\begin{itemize}
					\item визуално да разграничава нормалните от VIP обявите
					\item показва контактите на брокера, свързан с обявата
					\item предлага възможност за автоматично преизчисление на цената в алтернативна валута (BGN, USD, EUR).				Системата трябва да взима автоматично курса на БНБ.
					\item автоматично изчислява цена на квадратен метър (в текущо избраната валута)
					\item показва местоположението на обекта чрез Google Maps, на базата на GPS координатите му
					\item автоматично да преоразмерява снимките и скиците в thumbnail-и, които да показва, както и да дава възможност за разглеждане на оригиналните изображения			
				\end{itemize}
				
				Ако разглеждащият обявата е брокер или администратор, системата трябва да показва и информацията за наемодателя/продавача, както и детайлното местоположение на имота.
				
				Под детайлите на обявата системата трябва да предлага контактна форма за изпращане на съобщение до брокера. Формата има задължителни (име, email за обратна връзка, поле за свободен текст) и незадължителни (телефон за връзка) полета. Ако текущият потребител е регистриран, полетата email и телефон се попълват автоматично от системата с тези от профила на потребителя, като системата осигурява възможност на потребителя да ги редактира преди изпращане на съобщението. Изпращането на съобщения е еднопосочно - от потребители към брокер, и няма възможност за отговор през системата. Копие от всички изпратени през формата за контакт съобщения се изпраща до администратора.
			}

			\item {Търсене
			
				Потребителите да могат да търсят обяви в системата по текст и/или зададени филтри. Филтри могат да бъдат всички характеристични полета на обявата. 
				
				\emph{Забележка: някои видове характеристики са валидни само за някои видове имоти.}

				В резултата от търсенето, обявите тип VIP се показват с приоритет (преди останалите), като се разграничават от тях визуално. Няма ограничение за броя VIP обяви, които да излизат на страница.
			}
			
			\item {Чат
			
				Системата трябва да предоставя функционалност на нерегистрираните и регистрираните потребители да провеждат текстова комуникация в реално време с брокер. Потребителите могат да задават въпроси в чата, а брокерите, които са на работа наблюдават въпросите и отговарят на тях, като комуникацията е двустранна. Брокерите виждат само имената и въпросите на потребителите. Анонимните потребители трябва да предоставят име, което да използват в чата. Системата трябва да запазва историята на чат сесиите на регистираните потребители, и да им предоставя възможност да я разглеждат.
			}


			\item {Одит лог

				Системата трябва да поддържа одит лог, в който да записва всички действия на регистрирани потребители във системата. Логовете съдържат:
				\begin{itemize}
					\item дата и час на събитието
					\item извършител (username)
					\item от къде е извършено действието (IP адрес)
					\item тип на действието
					\item субект на действието
				\end{itemize}
  				Записват се всички промени върху потребителските профили и обявите. 	Одит лога може да се чете само от одиторският акаунт, и не може да се променя от никого.
			}
			
		\end{enumerate}
		
	}
	\item {Нефункционални изисквания
	
		Системата трябва да:
		\begin{itemize}
			\item е достъпна през уеб интерфейс
			\item поддържа около 100-150 посещения на ден
			\item поддържа 1000 регистрирани потребителя
			\item поддържа 400 активни обяви
			\item поддържа 10000 обяви общо (активни и неактивни)
			\item осигурява резултат при търсене до 3 секунди
			\item осигурява зареждане на страница за до 1-1.5 секунди
			\item поддържа паролите в криптиран вид
			\item може да работи върху GNU/Linux операционна система
			\item може да е в експлоатация 10 години
			\item има автоматичен онлайн бекъп не по-рядко от веднъж на 24 часа
		\end{itemize}
		
		Допустимият downtime на системата е не повече от 48 часа общо на година.
	}
	
	\item {Интеграция с външни системи

		Системата трябва да предлага възможност на потребителите (регистирани и нерегистрирани) да споделят дадена обява в социалните мрежи (Facebook, Twitter, Google+).

		Системата трябва да може да показва местоположението на даден имот в Google Maps на базата на GPS координатите в обявата.

		Системата трябва да поддържа възможност за Google Ads.

		Системата трябва да може автоматично да взима валутните курсове (BGN/USD/EUR) от БНБ.	
	}
	
	\item {Неизяснени въпроси
		\begin{enumerate}[1.]
			\item Могат ли брокерите да разглеждат скритите данни в обявите на други брокери?
			\item Какви права има администраторът върху обявите (създаване, редактиране, изтриване, статуси, брокер)?
			\item Какви права имат ролите брокер и администратор върху \emph{акаунтите} на останалите потребители?
			\item Какво е ограничението за размер на полето за свободен текст във формата за контакт?
			\item Какво могат да правят ролите брокер и администратор с получените съобщения през контакт формата?
			\item Детайлно описание на системата за чат -- Когато потребител зададе въпрос трябва ли да избере конкретен брокер?
			\item Детайлно описание за \emph{търсенето по текст} на обяви
			\item Изисквания към живота на одит лога
			\item Детайлно описание на изискванията към поддръжката на криптираните пароли
			\item Администраторът има ли правата на регистриран или нерегистриран потребител?
			\item При редактиране/премахване на обява от администратор ще има ли уведомление към съответния брокер, който я е вписал?
			\item В случай, че администратор премахне статус „брокер“ на потребител, какво се случва с вписаните от него обяви? Автоматична промяна на статуса на обявите към неактивни, друго?
			\item Описание на въвеждането/редактирането на обява от брокер
			\item Изисквания към потребителският интерфейс
			\item Изисквания към автоматично изтичане на сесия при неактивност
			\item Освен промените по потребителските акаунти и обявите, одит лога съдържа ли други типове действия?
			\item Промяната на данни в акаунт записва ли в одит лога старата и новата стойност?
			\item Това ли е пълният списък на характеристични полета за обявите?
		\end{enumerate}
	}
\end{enumerate}

\end{document}